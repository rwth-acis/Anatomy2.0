\documentclass[compress,xcolor=table]{beamer}
% add handout to optional args for handout version

\usepackage{beamerthemesplit}
\usepackage[utf8]{inputenc}
\usepackage[german]{babel}
\usepackage{german}
\usepackage{graphicx}
\usepackage{subfigure}
\usepackage{epsfig}
\usepackage{tabularx}
\usepackage{latexsym}
\usepackage{url}
\usepackage{tikz}
\usepackage{multimedia}
\usepackage{xcolor}
\usepackage{eso-pic}
\usepackage{color}
\usepackage{type1cm}
\usepackage{listings}
\usepackage{verbatim}
\usepackage{colortbl}
\usepackage{psfrag}
\usepackage{xifthen}
\usepackage[absolute,overlay]{textpos}
\usepackage{palatino}
\usepackage{appendixnumberbeamer}

%% Mulberry Color for highlighting
\definecolor{Mulberry}{cmyk}{0.34,0.90,0,0.02}
\definecolor{Lavender}{cmyk}{0,0.48,0,0}
\definecolor{Melon}{cmyk}{0,0.46,0.50,0}
\definecolor{Peach}{cmyk}{0,0.50,0.70,0}
\definecolor{RedOrange}{cmyk}{0,0.77,0.87,0}
\definecolor{BrickRed}{cmyk}{0,0.89,0.94,0.28}
\definecolor{Mahogany}{cmyk}{0,0.85,0.87,0.35}
\definecolor{BurntOrange}{cmyk}{0,0.51,1,0}
\definecolor{BitterSweet}{cmyk}{0,0.75,1,0.24}
\definecolor{FawkesRed}{rgb}{0.53,0,0}
\definecolor{RosBlue}{rgb}{0.19,0.25,0.38}


\mode<presentation>
{
  \usetheme{FawkesSimple}

  % to hide nav bar uncomment this line
  \setbeamertemplate{navigation symbols}{}

  %\setbeamercovered{transparent}
  \setbeamercovered{%
    again covered={\opaqueness<1->{40}}
  }
}

% Usage notes for handout version:
% Compile the beamer version immediately before you build the handout version,
% otherwise page numbers etc. will be wrong! The .aux files are *not* updated
% in handout mode, see PGF Manual for details why this is necessary.
% Comment out below one of the two pgfuselayout lines for either 2 or 4 slides
% per page. To have a very light grey background uncomment the background canvas
% color line. The logical page options are used to draw borders around each
% slide.
\mode<handout>
{
  \usetheme{Fawkes}

  % to hide nav bar uncomment this line
  \setbeamertemplate{navigation symbols}{}

  %\setbeamercovered{transparent}
  \setbeamercovered{%
    again covered={\opaqueness<1->{40}}
  }

  % Very slight grey background, can be used instead of borders
  %\setbeamercolor{background canvas}{bg=black!5}

  \usepackage{pgfpages}
  \pgfpagesuselayout{4 on 1}[a4paper,border shrink=5mm,landscape]
  %\pgfpagesuselayout{2 on 1}[a4paper,border shrink=5mm]

  \pgfpageslogicalpageoptions{1}{border code=\pgfstroke}
  \pgfpageslogicalpageoptions{2}{border code=\pgfstroke}
  \pgfpageslogicalpageoptions{3}{border code=\pgfstroke}
  \pgfpageslogicalpageoptions{4}{border code=\pgfstroke}
  \nofiles
}


% Declare layers
\pgfdeclarelayer{background}
\pgfsetlayers{background,main} 

% Load PGF libraries
\usetikzlibrary{patterns}
\usetikzlibrary{arrows}
\usetikzlibrary{topaths}
\usetikzlibrary{snakes}
\usetikzlibrary{calc}
\usetikzlibrary{positioning}
\usetikzlibrary{shadows}
\usetikzlibrary{shapes.multipart}


% set lengths for textpos package
\setlength{\TPHorizModule}{10mm}
\setlength{\TPVertModule}{\TPHorizModule}
\textblockorigin{8mm}{16mm} % start everything near the top-left corner
\setbeamercolor{textblock color}{fg=blue!50,bg=white}

\urlstyle{sf}
\urldef{\projecturl}\url{}

%\pgfdeclareimage[width=4cm]{logo-big}{syslife/logo_big}

\institute{%
  %\vspace{1cm}
  \begin{minipage}{\textwidth}\centering
  \includegraphics[height=0.6cm]{images/rwth-logo}
  \end{minipage}

  \bigskip

  \begin{minipage}{\textwidth}\centering
  \textcolor{FawkesBrown}
  \projecturl
  \end{minipage}
  %\vspace{-1.5cm}
    %  \end{column}
    %\end{columns}
  %\end{minipage}
}


%\titlegraphic{\pgfuseimage{logo-big}}

% \AtBeginPart{\frame{\partpage}}

%numbers=left, numberstyle=\tiny, stepnumber=2, numbersep=5pt
\lstset{language=[GNU]C++,
        basicstyle=\small,
        escapeinside={/*(*/}{/*)*/},
        breaklines=true,
        showstringspaces=false
        }

\lstdefinelanguage{JavaScript}{
  keywords={typeof, new, true, false, catch, function, return, null, catch, switch, var, if, in, while, do, else, case, break},
  keywordstyle=\color{blue}\bfseries,
  ndkeywords={class, export, boolean, throw, implements, import}, %, this
  ndkeywordstyle=\color{darkgray}\bfseries,
  identifierstyle=\color{black},
  sensitive=false,
  comment=[l]{//},
  morecomment=[s]{/*}{*/},
  commentstyle=\color{purple}\ttfamily,
  stringstyle=\color{red}\ttfamily,
  morestring=[b]',
  morestring=[b]"
}

\lstdefinestyle{JSON}
{
  language=JavaScript,
  morekeywords={interface,field,message,comment},
  basicstyle=\footnotesize\ttfamily\vspace{0.2cm},
  breaklines=true,
  showstringspaces=false,
  %keywordstyle=\bfseries,
  keywordstyle=\color{Mulberry},
  frame=lines,
  belowcaptionskip=8pt,
  emphstyle=\itshape,
  numbers=left,
  stepnumber=1,
  backgroundcolor=\color{blue!10},
  rulecolor=\color{blue!50},
  fillcolor=\color{blue!20},
  framexleftmargin=16pt,
  xleftmargin=16pt,
  %stringstyle=\color{BitterSweet},
  stringstyle=\color{BrickRed},
  commentstyle=\color{BrickRed},
  escapechar=\%
  % emph={getup, servo, depends_skills},
  %emphstyle=\underbar,
  %numbers=left,
  %stepnumber=1,
  %%stringstyle=\ttfamily, % typewriter type for strings
}

\lstdefinestyle{SmallJSON}{
  style=JSON,
  basicstyle=\ttfamily\footnotesize,
  numbersep=6pt,
}
\lstdefinestyle{ReallySmallJSON}{
  style=JSON,
  basicstyle=\ttfamily\tiny,
  numbersep=5pt,
}


% Listings stuff
\lstdefinelanguage{Lua}
{
  morekeywords={and,break,do,else,elseif,end,false,for,function,
                if,in,local,nil,not,or,repeat,return,then,true,until,while},
  sensitive=true,
  morecomment=[l]{--},
  morecomment=[s]{--[[}{--]]},
  morestring=[b]{"},
  morestring=[s]{[==[}{]==]},
}

% default style
\lstdefinestyle{Lua}
{
  language=Lua,
  basicstyle=\ttfamily,
  breaklines=true,
  showstringspaces=false,
  %keywordstyle=\bfseries,
  keywordstyle=\color{Mulberry},
  %frame=lines,
  %belowcaptionskip=8pt,
  emphstyle=\itshape,
  %numbers=left,
  stepnumber=1,
  %backgroundcolor=\color{blue!10},
  rulecolor=\color{blue!50},
  fillcolor=\color{blue!20},
  %framexleftmargin=18pt,
  %xleftmargin=18pt,
  stringstyle=\color{BitterSweet},
  %stringstyle=\color{BrickRed},
  commentstyle=\color{BrickRed},
  escapechar=\%
  % emph={getup, servo, depends_skills},
  %emphstyle=\underbar,
  %numbers=left,
  %stepnumber=1,
  %%stringstyle=\ttfamily, % typewriter type for strings
}
\lstdefinestyle{SmallLua}{
  style=Lua,
  basicstyle=\ttfamily\footnotesize,
  numbersep=6pt,
}
\lstdefinestyle{ReallySmallLua}{
  style=Lua,
  basicstyle=\ttfamily\tiny,
  numbersep=5pt,
}

% Default is Lua
\lstset{style=Lua}


% Hyphenation of words with hyphen
\def\hyph{-\penalty0\hskip0pt\relax}


% define an anchor in the frame
\newcommand{\tikzref}[1]{%
  \tikz[remember picture]{%
    \coordinate (#1) at (0,0.5ex);%
  }%
}%


%%% Local Variables: 
%%% mode: latex
%%% TeX-master: "iros2012-robodb"
%%% End: 


\title[Collaborative 3D Model Viewing on the Web]{Collaborative 3D Model Viewing on the Web\\ First Review}
\author[Group 2]{%
  Alexandra Wörner, Dominik Studer, Ali Demiralp\\ Dev Sharma and Frederik Zwilling \\
  \bigskip
  {\scriptsize Group 2}
}

\date[Nov 14th 2014 @ HENM 2014]{Nov 14th 2014 -- First Review HENM}

\begin{document}

\frame[plain]{\titlepage}
\addtocounter{framenumber}{-1}

\begin{frame}
  \frametitle{Overview}
  \tableofcontents[hideallsubsections]
\end{frame}
%\addtocounter{framenumber}{-1}

\section{Project Description}

\begin{frame}
  \frametitle{At a Glance}
  \begin{columns}
    \begin{column}{0.8\textwidth}
      \begin{description}[]
        \item[Why Collaborative 3D Model Viewing?] \hfill \\
        \begin{itemize}
          \item Teaching and learning from 3D models more comprehensive than from 2D drawing
          \item Access to real objects costly or limited
          \item Collaborative viewing allows new, interactive learning methods
        \end{itemize}

        \bigskip
        \item[Our Approach] \hfill \\
          \begin{itemize}
            \item Upload models (especially from 3D scanning) into a database
            \item View models in a browser on different devices
            \item Manipulate view on all devices at the same time
        \end{itemize}
      \end{description}
    \end{column}

    \begin{column}{0.2\textwidth}
      \includegraphics[width=.9\textwidth]{images/skull}\\
      \bigskip
      \includegraphics[width=.9\textwidth]{images/nefertiti}
    \end{column}
  \end{columns}
\end{frame}

\section{Technology survey}

\begin{frame}
  \frametitle{ROLE}
      \begin{description}[]
        \item[ROLE (Responsive Open Learning Environments)] \hfill \\
        \begin{itemize}
          \item Learning environment in the web
          \item Personalized with widgets\\ $\Rightarrow$ synergies
          \item Collaborative learning\\ $\Rightarrow$ inter-device-communication
        \end{itemize}
      \end{description}
\end{frame}


\section{Market analysis}

\begin{frame}
  \frametitle{Market analysis}
 \begin{columns}
    \begin{column}{0.6\textwidth}
      \begin{description}[]
        \item[Industrial Diversity] \hfill \\
        \begin{itemize}
          \item Medical industry, Architecture, Computer Graphics, TV, movies, video games.
          \item Educational sector.
        \end{itemize}

        \item[Wide User base] \hfill \\
          \begin{itemize}
            \item Everyone with a mobile device or desktop computer that supports the used technology.
            \item 75\% of desktop computers and 50\% of mobile devices are equipped with WebGL-enabled browsers.
        \end{itemize}
      \end{description}
    \end{column}

    \begin{column}{0.5\textwidth}
      \includegraphics[width=.9\textwidth]{images/devices.png}\\
    \end{column}
  \end{columns}
\end{frame}

\begin{frame}
  \frametitle{Market analysis}
      \begin{description}[]
        \item[Industry cost structure] \hfill \\
        \begin{itemize}
          \item Scanning equipment is expensive, but will become more affordable.
          \item Standardized web technologies and SDKs.
        \end{itemize}

        \item[Distribution Channels] \hfill \\
          \begin{itemize}
            \item Medical Faculty at RWTH Aachen.
            \item Expand to more medical faculties in Germany and  broaden the scope of application.
        \end{itemize}

\item[Key Success Factors] \hfill \\
          \begin{itemize}
            \item Adjust to new technology/developments.
           \item Abilities in programming.
	\item Building reputation.
\end{itemize}
\end{description}
\end{frame}

\section{Gantt Chart}

\begin{frame}
  \frametitle{Gantt Chart}
A Gantt chart is a bar chart that shows the tasks of a project, when each task must take place and how long each will take.  
  \begin{description}[]
	\item[Conventions] \hfill \\
	\begin{itemize}
	\item Try to keep it simple.
          \item Highlight milestones/reviews.
        \item Assign different colours to team members .
        \item Indicate the (temporal) dependency of one task from another with arrows.
\end{itemize}
 \end{description}
\end{frame}

\section{Scrum, Team process}

\begin{frame}
  \frametitle{Scrum}
\end{frame}

\begin{frame}
  \frametitle{Reflection on team process}
\end{frame}

\section{Conclusion}

\begin{frame}
  \frametitle{Conclusion and next steps}
\end{frame}

\end{document}
